\documentclass[10pt,letterpaper]{article}
\usepackage[english]{babel}
\usepackage{natbib}
\usepackage{url}
\usepackage[utf8x]{inputenc}
\usepackage{amsmath}
\usepackage{graphicx}
\graphicspath{{images/}}
\usepackage{parskip}
\usepackage{fancyhdr}
\usepackage{vmargin}
\usepackage{listings}
\usepackage[ruled]{algorithm2e}
\usepackage{program}
\usepackage{multicol}

\title{Hacerle la porta ql}								% Title
\author{Sergio Salinas Fern\'andez \\
Danilo Abellá}								% Author
\date{\today}											% Date


\makeatletter
\let\thetitle\@title
\let\theauthor\@author

\let\thedate\@date
\makeatother

\pagestyle{fancy}
\chead{Complejidad}
\rhead{LCC}
\lhead{USACH}
%\cfoot{}

\begin{document}

%%%%%%%%%%%%%%%%%%%%%%%%%%%%%%%%%%%%%%%%%%%%%%%%%%%%%%%%%%%%%%%%%%%%%%%%%%%%%%%%%%%%%%%%%
\begin{titlepage}

\begin{center}

{\Large { Universidad Santiago de Chile\\Facultad de Ciencia\\Licenciatura en Ciencia de la Computación
 }}\\[5cm]



{\huge \textsc{Complejidad de Algoritmos}}\\[1.5cm]

{\Huge ``Algoritmos de
búsqueda y ordenamiento'' }\\[1.5cm]


{\LARGE 2\textsuperscript{o} Laboratorio}\\[8cm]

\hspace*{\fill} 
\end{center}
\begin{minipage}[l]{0.4\textwidth}
	\begin{flushleft}
	\linespread{1}
	\end{flushleft}
\end{minipage}
\begin{minipage}[l]{0.6\textwidth}

	\begin{flushright}
		
		\textbf{Datos Informe}\\[0.1cm]
		\large \textbf{Nombre alumnos:} Sergio Salinas\\
		\large Danilo Abellá\\
\large \textbf{Nombre profesor:} Nicolas Thériault \\
	\end{flushright}
\end{minipage}
\end{titlepage}
\newpage
\newpage

%%%%%%%%%%%%%%%%%%%%%%%%%%%%%%%%%%%%%%%%%%%%%%%%%%%%%%%%%%%%%%%%%%%%%%%%%%%%%%%%%%%%%%%%%

\tableofcontents
\pagebreak

%%%%%%%%%%%%%%%%%%%%%%%%%%%%%%%%%%%%%%%%%%%%%%%%%%%%%%%%%%%%%%%%%%%%%%%%%%%%%%%%%%%%%%%%%

\newpage
\section{Problema 1}

\subsection{Solución Encontrada}

La solución que se encontró es que el número que más se repite es 5896 con 16 repitencias.


\subsection{Estrategia}

El algoritmo guarda todos los caracteres del archivo tex en un arreglo A y crea otro arreglo B de lleno de ceros que tiene 10000 espacios de memoria y se usa para almacenar las frecuencias que tiene cada número.

En un ciclo, se obtienen los primeros 4 dígitos de A y se usan como indice para acceder al espacio de memoria que corresponde a ese indice y aumenta la frecuencia de aparición de ese número en 1, luego se quita el primer elemento de A aumentando la posición del puntero en 1 y se vuelve al inicio del ciclo. Esto se repite hasta que se recorra A por completo.


Además hay otra variable que se encarga de guardar la mayor frecuencia durante el ciclo, para mostrar el número que más repite solo se busca en B el indice que tenga esa frecuencia.

\subsection{Justificación}

Se utiliza este método por su efectividad, en lo que es la memoria, el arreglo A es de largo n y el arreglo B siempre es constante ya que siempre tendrá 10000 espacios de memoria que son los necesarios para guardar la frecuencia de los números del 0 al 9999, por lo la memoria total utilizada es de n + 1000, por lo que es de orden O(n).

En lo que es la complejidad del algoritmo, es de orden O(n), esto es debido a que en el ciclo de aumentar la frecuencia de B dado un indice obtenido de A su operación básica se repite n veces, y en el ciclo de buscar el mayor número la operación comprar se repite 10000 veces, por lo que se tiene que la complejidad es n + 10000.

\newpage

\section{Problema 2}

\subsection{Solución Encontrada}

Como solución se tienen las siguientes secuencias de 4 “dígitos” (hexadigitos) consecutivos que se repiten en el número hexadecimal dado:

Secuencia Repetida - Frecuencia.

\begin{multicols}{5}
0006 -  2\\
0068 -  2\\
00BB -  2\\
00DF -  2\\
0115 -  2\\
011A -  3\\
013C -  2\\
01AB -  2\\
01AD -  2\\
03A1 -  2\\
0402 -  2\\
04C0 -  2\\
056A -  2\\
0584 -  2\\
061B -  2\\
0734 -  2\\
07EF -  2\\
0810 -  2\\
085F -  2\\
08BA -  2\\
0A12 -  2\\
0A47 -  2\\
0BA6 -  2\\
0BCB -  2\\
0DC7 -  2\\
0DF0 -  2\\
0F28 -  2\\
1018 -  2\\
105D -  2\\
10B3 -  2\\
1133 -  2\\
1156 -  2\\
1159 -  2\\
1183 -  2\\
11A1 -  2\\
1290 -  3\\
12B4 -  2\\
12DC -  2\\
133A -  2\\
13E0 -  2\\
1421 -  2\\
1462 -  2\\
14BA -  2\\
14CD -  2\\
14D9 -  2\\
14E5 -  3\\
153C -  2\\
1548 -  2\\
15D6 -  2\\
1612 -  2\\
1625 -  2\\
1636 -  2\\
16DF -  2\\
1811 -  2\\
183B -  2\\
1856 -  2\\
18B1 -  2\\
1936 -  2\\
1948 -  2\\
19EE -  2\\
1BD3 -  2\\
1C20 -  2\\
1C36 -  2\\
1C90 -  2\\
1CE6 -  2\\
1D00 -  2\\
1D39 -  2\\
1DC2 -  2\\
1DDF -  2\\
1E0A -  2\\
1E63 -  2\\
1E79 -  2\\
1E8A -  2\\
1EAF -  2\\
1EBD -  2\\
1F2E -  2\\
1F6C -  2\\
1F9B -  2\\
202E -  2\\
20AD -  2\\
20C8 -  2\\
2105 -  2\\
21E7 -  2\\
2299 -  2\\
22B6 -  2\\
2463 -  2\\
2466 -  2\\
2469 -  2\\
2547 -  2\\
25BD -  2\\
2623 -  2\\
2690 -  2\\
2796 -  2\\
27A1 -  2\\
2821 -  2\\
2850 -  2\\
2851 -  2\\
2858 -  2\\
2895 -  2\\
28AB -  2\\
299B -  2\\
299F -  2\\
2AB3 -  2\\
2ADA -  2\\
2B89 -  2\\
2B8B -  2\\
2D38 -  2\\
2D51 -  2\\
2DFD -  2\\
2E13 -  2\\
2E28 -  2\\
2E9C -  2\\
2EB3 -  2\\
2EF1 -  2\\
2EF6 -  2\\
2F32 -  2\\
30DC -  2\\
314E -  2\\
317C -  2\\
31EA -  2\\
31F6 -  3\\
320F -  2\\
33A3 -  2\\
33E8 -  2\\
3604 -  2\\
361D -  2\\
369B -  2\\
3707 -  2\\
3820 -  2\\
3822 -  2\\
38D8 -  2\\
392E -  2\\
396A -  2\\
3A3A -  2\\
3AE1 -  2\\
3B12 -  2\\
3B3E -  2\\
3C6F -  2\\
3D5A -  2\\
3D7C -  2\\
3E81 -  2\\
3E89 -  2\\
3FD6 -  2\\
4000 -  2\\
4152 -  2\\
4183 -  2\\
42EF -  2\\
42F5 -  2\\
42F6 -  2\\
4324 -  3\\
442E -  2\\
4574 -  2\\
4595 -  2\\
4614 -  2\\
46DB -  2\\
46F6 -  2\\
46FC -  2\\
479B -  2\\
4898 -  2\\
48E4 -  2\\
48FD -  2\\
49F1 -  2\\
4A41 -  2\\
4A99 -  2\\
4B27 -  2\\
4B47 -  2\\
4BA3 -  2\\
4BFB -  2\\
4CDD -  2\\
4E3D -  2\\
4E6C -  2\\
501A -  3\\
5056 -  2\\
5094 -  2\\
5133 -  2\\
525F -  2\\
5282 -  2\\
52DF -  2\\
52EC -  2\\
532E -  2\\
542F -  2\\
5470 -  2\\
5546 -  2\\
55AB -  2\\
55FD -  2\\
5605 -  2\\
562E -  2\\
5664 -  2\\
56E1 -  2\\
5748 -  3\\
5770 -  2\\
586E -  2\\
5882 -  2\\
593E -  2\\
59CB -  2\\
5AA5 -  2\\
5BDD -  2\\
5C02 -  2\\
5CB0 -  2\\
5CFA -  3\\
5D88 -  2\\
5EE3 -  2\\
5F01 -  2\\
5F79 -  2\\
5F95 -  2\\
5FEC -  2\\
602A -  2\\
6078 -  2\\
6085 -  2\\
612A -  2\\
614E -  2\\
618B -  2\\
6282 -  3\\
6293 -  2\\
62E9 -  2\\
6369 -  2\\
6382 -  2\\
6629 -  3\\
6636 -  2\\
6645 -  2\\
66A0 -  2\\
67BC -  2\\
6842 -  2\\
698D -  2\\
6A36 -  2\\
6A37 -  2\\
6A51 -  2\\
6B2A -  2\\
6B6A -  2\\
6DFF -  2\\
6E2F -  2\\
6E37 -  2\\
6EA6 -  2\\
6F47 -  2\\
6FB2 -  2\\
6FF3 -  2\\
7061 -  2\\
7080 -  2\\
7157 -  2\\
7216 -  2\\
724D -  2\\
74E6 -  2\\
7509 -  2\\
7560 -  2\\
75B1 -  2\\
7634 -  2\\
7711 -  2\\
771F -  2\\
7732 -  2\\
7808 -  2\\
78C1 -  2\\
7960 -  2\\
7A58 -  2\\
7B8E -  2\\
7B9F -  2\\
7CCF -  2\\
7CD3 -  2\\
7D9C -  2\\
7DA8 -  2\\
7DD1 -  2\\
7E6A -  2\\
800B -  2\\
80BB -  2\\
8128 -  2\\
8129 -  2\\
8162 -  2\\
82EF -  2\\
8346 -  2\\
83B5 -  2\\
84A5 -  2\\
8507 -  2\\
8509 -  2\\
8556 -  2\\
858E -  2\\
85A3 -  2\\
85F0 -  2\\
86E3 -  2\\
8740 -  2\\
8839 -  2\\
8986 -  2\\
8A0B -  2\\
8B1D -  2\\
8B38 -  2\\
8D01 -  2\\
8D40 -  2\\
8DB3 -  3\\
8F48 -  2\\
8FD2 -  2\\
9045 -  2\\
924A -  2\\
9317 -  2\\
9361 -  2\\
93D5 -  2\\
93E8 -  2\\
94B7 -  3\\
94C2 -  2\\
9586 -  2\\
9662 -  2\\
96A2 -  2\\
9802 -  2\\
991B -  2\\
993B -  2\\
9A45 -  2\\
9A53 -  2\\
9B04 -  2\\
9B07 -  2\\
9B6F -  2\\
9C2B -  2\\
9C60 -  2\\
9D65 -  2\\
9DCB -  2\\
9E15 -  2\\
9E86 -  2\\
9F25 -  2\\
9F8D -  2\\
9FB4 -  2\\
A0C1 -  2\\
A0E2 -  2\\
A124 -  2\\
A14D -  2\\
A161 -  2\\
A1D4 -  2\\
A285 -  2\\
A323 -  2\\
A366 -  2\\
A37F -  2\\
A3CB -  2\\
A476 -  2\\
A58C -  2\\
A65C -  2\\
A7A9 -  2\\
A835 -  2\\
A995 -  2\\
A9BE -  2\\
ABEA -  2\\
ABFC -  2\\
AC71 -  2\\
ACE1 -  2\\
ADA3 -  3\\
AE4D -  2\\
AFD6 -  2\\
B023 -  2\\
B03A -  2\\
B155 -  2\\
B1DC -  2\\
B21A -  2\\
B266 -  3\\
B277 -  2\\
B27B -  2\\
B285 -  2\\
B2F3 -  2\\
B331 -  2\\
B38E -  2\\
B3B1 -  2\\
B3EE -  2\\
B457 -  2\\
B513 -  2\\
B557 -  2\\
B5FA -  2\\
B6A3 -  2\\
B6C1 -  2\\
B750 -  2\\
B7D9 -  2\\
B7E3 -  2\\
B812 -  2\\
B820 -  2\\
B8B5 -  2\\
B93D -  2\\
B9D3 -  2\\
BA33 -  2\\
BA99 -  2\\
BA9B -  3\\
BB3A -  2\\
BB56 -  2\\
BBCA -  2\\
BC2A -  2\\
BC9B -  2\\
BD31 -  2\\
BE32 -  2\\
BF00 -  2\\
BF2C -  2\\
C115 -  2\\
C25A -  2\\
C277 -  3\\
C2DA -  2\\
C2DD -  2\\
C372 -  2\\
C4BF -  2\\
C511 -  2\\
C6E8 -  2\\
C902 -  2\\
C904 -  2\\
C946 -  2\\
C9E0 -  2\\
CA92 -  2\\
CB06 -  2\\
CBEE -  2\\
CCC8 -  2\\
CE0B -  2\\
CE77 -  2\\
CF0B -  2\\
D04C -  2\\
D1B5 -  2\\
D1CF -  2\\
D1D0 -  2\\
D519 -  2\\
D586 -  2\\
D65F -  2\\
D6A1 -  2\\
D6EB -  2\\
D83D -  2\\
D8FE -  2\\
D915 -  2\\
D993 -  2\\
D9B9 -  2\\
DA2F -  2\\
DAE9 -  2\\
DC09 -  2\\
DC14 -  2\\
DC25 -  2\\
DD57 -  2\\
DDF8 -  2\\
DF01 -  2\\
DFA1 -  2\\
DFA6 -  2\\
DFF8 -  2\\
E03A -  2\\
E0E1 -  2\\
E14E -  2\\
E153 -  2\\
E169 -  2\\
E1B0 -  2\\
E1E7 -  2\\
E1F9 -  2\\
E305 -  2\\
E39F -  2\\
E3D0 -  2\\
E3D3 -  2\\
E4C6 -  2\\
E4D6 -  2\\
E593 -  2\\
E60B -  2\\
E647 -  2\\
E69F -  2\\
E6AD -  2\\
E6BA -  2\\
E6C5 -  2\\
E799 -  2\\
E8A3 -  2\\
E8AE -  2\\
E8EF -  2\\
E9DB -  2\\
EB26 -  2\\
EB65 -  2\\
EC6E -  2\\
ECC8 -  2\\
ECF1 -  2\\
ED93 -  2\\
EDFA -  2\\
EE30 -  2\\
EE60 -  2\\
EECC -  2\\
EF6A -  2\\
EF7D -  2\\
F011 -  2\\
F017 -  2\\
F01C -  2\\
F050 -  2\\
F06A -  2\\
F092 -  2\\
F0BD -  2\\
F0CA -  2\\
F0F7 -  2\\
F19B -  2\\
F2B8 -  2\\
F442 -  2\\
F5EB -  2\\
F708 -  2\\
F724 -  2\\
F728 -  2\\
F74B -  2\\
F74E -  2\\
F7DA -  2\\
F845 -  2\\
F89E -  2\\
F8E7 -  2\\
F8E8 -  2\\
F953 -  2\\
F95E -  2\\
FA3D -  2\\
FAD5 -  2\\
FBC9 -  2\\
FCED -  2\\
FD2C -  2\\
FE6E -  2\\
FF8E -  2\\
FFA3 -  2\\
FFEA -  2\\

\end{multicols}


\subsection{Estrategia}

Se utiliza un arreglo "a" para guardar todos los dígitos del archivo tex y "hexarray" (solo con ceros y 10000 en espacio de memoria) para guardar las secuencias de 4 dígitos hexadecimeales.
Nota: Para trabajar con números hexadecimales se utilizó la función "strtol(char, NULL, 16)". 

Luego se recorrió el arreglo "hexarray" ordenando todas las secuencias de 4 dígitos con el método de ordenamiento por inserción.

Una vez ordenado de menor a mayor, el arreglo es recorrido y se van mostrando las secuencias que se repiten al menos 1 vez con su respectiva frecuencia.

\subsection{Justificación}

Se decide crear una lista ordenada con todos los números de 4 dígitos del archivo tex debido a su facilidad de uso y porque el coste en memoria de este arreglo es de n-4. La complejidad de obtener todos los números de 4 dígitos del arreglo con los datos del archivo es de n-4.

Se decide usar el método de inserción dado a sus bajos requerimientos de memoria y estabilidad (nunca intercambia registros con claves iguales). Pese a su lento proceso y la numerosas comparaciones necesarias a realizar, muestra un comportamiento razonablemente bien en gran cantidad de situaciones, y esta no es la excepción, además que ser de fácil implementación.

El método de inserción tiene como complejidad en el peor de los casos O($n^2$) y en el mejor de los casos O(n) (o sea, cuando el arreglo está ordenado), y el recorrido guardar las secuencias de 4 dígitos hexadecimeales una es de n-4.

Por lo que la complejidad del algoritmo sería: 

En el mejor caso: n + (n-4)
\\
En el peor caso: $n^2$ + (n-4)


Y su uso de memoria total es de 2n-4, debido a que n es lo que cuesta el arreglo con los dígitos de un archivo, n-4 el arreglo con todos los números de 4 dígitos y el insertion sort solo requiere O(1) de espacio adicional.


\newpage
\section{Problema 3}

\subsection{Solución Encontrada}

$Archivo_3.tex$ no arroja ningún resultado, por lo que no hay ninguna secuencia de 6 dígitos repetidas según el corte que se le dio.

\subsection{Estrategia}


Se uso un método similar al segundo ejercicio, se guarda la data del texto $Archivo_3.tex$ en un arreglo A de largo n, este arreglo se recorre $\frac{n}{6}$ veces extrayendo en segmentos de 6 dígitos cada número del arreglo y guardandolos en un nuevo arreglo B en forma de INT.

Este arreglo B se ordena usando el método de ordenamiento quicksort, de esta forma el arreglo B tiene los elementos repetidos uno al lado del otro, por último se recorre el arreglo contando la frecuencia con la que aparece cada elemento repetido de manera seguida y los que tengan frecuencia mayor a uno se muestran.


\subsection{Justificación}

Se utilizo este método debido a que su memoria es lineal, solo ocupa dos arreglos de largo n y $\frac{n}{6}$, el quicksort ocupa $\log n$ de memoria, por lo que ocupa $\frac{7n}{6} + \log n$ de memoria.

Además de su rapidez, dividir el arreglo en secciones de 6 tiene complejidad de $\frac{n}{6}$ y ordenar el arreglo con el método quicksort tiene una complejidad de $n \log n $, por lo que su complejidad final es de $\frac{n}{6} + n \log n $, haciendo que el algoritmo sea de complejidad lineal logarítmica.

Se usa el algoritmo quicksort debido a su rapidez y que ser el largo del arreglo par es más seguro que el pivote elegido sea el correcto.


\end{document}
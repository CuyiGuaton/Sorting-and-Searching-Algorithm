\documentclass[10pt,letterpaper]{article}
\usepackage[english]{babel}
\usepackage{natbib}
\usepackage{url}
\usepackage[utf8x]{inputenc}
\usepackage{amsmath}
\usepackage{graphicx}
\graphicspath{{images/}}
\usepackage{parskip}
\usepackage{fancyhdr}
\usepackage{vmargin}
\usepackage{listings}
\usepackage[ruled]{algorithm2e}
\usepackage{program}

\title{Hacerle la porta ql}								% Title
\author{Sergio Salinas Fern\'andez \\
Danilo Abellá}								% Author
\date{\today}											% Date


\makeatletter
\let\thetitle\@title
\let\theauthor\@author

\let\thedate\@date
\makeatother

\pagestyle{fancy}
\chead{Complejidad}
\rhead{LCC}
\lhead{USACH}
%\cfoot{}

\begin{document}

%%%%%%%%%%%%%%%%%%%%%%%%%%%%%%%%%%%%%%%%%%%%%%%%%%%%%%%%%%%%%%%%%%%%%%%%%%%%%%%%%%%%%%%%%
\begin{titlepage}

\begin{center}

{\Large { Universidad Santiago de Chile\\Facultad de Ciencia\\Licenciatura en Ciencia de la Computación
 }}\\[5cm]



{\huge \textsc{Complejidad de Algoritmos}}\\[1.5cm]

{\Huge ``Algoritmos de
búsqueda y ordenamiento'' }\\[1.5cm]


{\LARGE 2\textsuperscript{o} Laboratorio}\\[8cm]

\hspace*{\fill} 
\end{center}
\begin{minipage}[l]{0.4\textwidth}
	\begin{flushleft}
	\linespread{1}
	\end{flushleft}
\end{minipage}
\begin{minipage}[l]{0.6\textwidth}

	\begin{flushright}
		
		\textbf{Datos Informe}\\[0.1cm]
		\large \textbf{Nombre alumnos:} Sergio Salinas\\
		\large Danilo Abellá\\
\large \textbf{Nombre profesor:} Dino Araya\\
	\end{flushright}
\end{minipage}
\end{titlepage}
\newpage
\newpage

%%%%%%%%%%%%%%%%%%%%%%%%%%%%%%%%%%%%%%%%%%%%%%%%%%%%%%%%%%%%%%%%%%%%%%%%%%%%%%%%%%%%%%%%%

\tableofcontents
\pagebreak

%%%%%%%%%%%%%%%%%%%%%%%%%%%%%%%%%%%%%%%%%%%%%%%%%%%%%%%%%%%%%%%%%%%%%%%%%%%%%%%%%%%%%%%%%

\newpage
\section{Problema 1}

\subsection{Solución Encontrada}

La solución que se encontró es que el número que más se repite es 5896 con 16 repitencias.


\subsection{Estrategia}

El algoritmo guarda todos los caracteres del archivo tex en un arreglo A y crea otro arreglo B de lleno de ceros que tiene 10000 espacios de memoria y se usa para almacenar las frecuencias que tiene cada número.

En un ciclo, se obtienen los primeros 4 dígitos de A y se usan como indice para acceder al espacio de memoria que corresponde a ese indice y aumenta la frecuencia de aparición de ese número en 1, luego se quita el primer elemento de A aumentando la posición del puntero en 1 y se vuelve al inicio del ciclo. Esto se repite hasta que se recorra A por completo.


Además hay otra variable que se encarga de guardar la mayor frecuencia durante el ciclo, para mostrar el número que más repite solo se busca en B el indice que tenga esa frecuencia.

\subsection{Justificación}

Se utiliza este método por su efectividad, en lo que es la memoria, el arreglo A es de largo n y el arreglo B siempre es constante ya que siempre tendrá 10000 espacios de memoria que son los necesarios para guardar la frecuencia de los números del 0 al 9999, por lo la memoria total utilizada es de n + 1000, por lo que es de orden O(n).

En lo que es la complejidad del algoritmo, es de orden O(n), esto es debido a que en el ciclo de aumentar la frecuencia de B dado un indice obtenido de A su operación básica se repite n veces, y en el ciclo de buscar el mayor número la operación comprar se repite 10000 veces, por lo que se tiene que la complejidad es n + 10000.

\newpage

\section{Problema 2}

\subsection{Solución Encontrada}

Como solución se tienen las siguientes secuencias de 4 “dígitos” (hexadigitos) consecutivos que se repiten en el número hexadecimal dado:

n0006-n0068-n00BB-n00DF-n0115-n011A-n013C-n01AB-n01AD-n03A1-n0402-n04C0-n056A-n0584-n061B-n0734-n07EF-n0810-n085F-n08BA-n0A12-n0A47-n0BA6-n0BCB-n0DC7-n0DF0-n0F28-n1018-n105D-n10B3-n1133-n1156-n1159-n1183-n11A1-n1290-n12B4-n12DC-n133A-n13E0-n1421-n1462-n14BA-n14CD-n14D9-n14E5-n153C-n1548-n15D6-n1612-n1625-n1636-n16DF-n1811-n183B-n1856-n18B1-n1936-n1948-n19EE-n1BD3-n1C20-n1C36-n1C90-n1CE6-n1D00-n1D39-n1DC2-n1DDF-n1E0A-n1E63-n1E79-n1E8A-n1EAF-n1EBD-n1F2E-n1F6C-n1F9B-n202E-n20AD-n20C8-n2105-n21E7-n2299-n22B6-n2463-n2466-n2469-n2547-n25BD-n2623-n2690-n2796-n27A1-n2821-n2850-n2851-n2858-n2895-n28AB-n299B-n299F-n2AB3-n2ADA-n2B89-n2B8B-n2D38-n2D51-n2DFD-n2E13-n2E28-n2E9C-n2EB3-n2EF1-n2EF6-n2F32-n30DC-n314E-n317C-n31EA-n31F6-n320F-n33A3-n33E8-n3604-n361D-n369B-n3707-n3820-n3822-n38D8-n392E-n396A-n3A3A-n3AE1-n3B12-n3B3E-n3C6F-n3D5A-n3D7C-n3E81-n3E89-n3FD6-n4000-n4152-n4183-n42EF-n42F5-n42F6-n4324-n442E-n4574-n4595-n4614-n46DB-n46F6-n46FC-n479B-n4898-n48E4-n48FD-n49F1-n4A41-n4A99-n4B27-n4B47-n4BA3-n4BFB-n4CDD-n4E3D-n4E6C-n501A-n5056-n5094-n5133-n525F-n5282-n52DF-n52EC-n532E-n542F-n5470-n5546-n55AB-n55FD-n5605-n562E-n5664-n56E1-n5748-n5770-n586E-n5882-n593E-n59CB-n5AA5-n5BDD-n5C02-n5CB0-n5CFA-n5D88-n5EE3-n5F01-n5F79-n5F95-n5FEC-n602A-n6078-n6085-n612A-n614E-n618B-n6282-n6293-n62E9-n6369-n6382-n6629-n6636-n6645-n66A0-n67BC-n6842-n698D-n6A36-n6A37-n6A51-n6B2A-n6B6A-n6DFF-n6E2F-n6E37-n6EA6-n6F47-n6FB2-n6FF3-n7061-n7080-n7157-n7216-n724D-n74E6-n7509-n7560-n75B1-n7634-n7711-n771F-n7732-n7808-n78C1-n7960-n7A58-n7B8E-n7B9F-n7CCF-n7CD3-n7D9C-n7DA8-n7DD1-n7E6A-n800B-n80BB-n8128-n8129-n8162-n82EF-n8346-n83B5-n84A5-n8507-n8509-n8556-n858E-n85A3-n85F0-n86E3-n8740-n8839-n8986-n8A0B-n8B1D-n8B38-n8D01-n8D40-n8DB3-n8F48-n8FD2-n9045-n924A-n9317-n9361-n93D5-n93E8-n94B7-n94C2-n9586-n9662-n96A2-n9802-n991B-n993B-n9A45-n9A53-n9B04-n9B07-n9B6F-n9C2B-n9C60-n9D65-n9DCB-n9E15-n9E86-n9F25-n9F8D-n9FB4-nA0C1-nA0E2-nA124-nA14D-nA161-nA1D4-nA285-nA323-nA366-nA37F-nA3CB-nA476-nA58C-nA65C-nA7A9-nA835-nA995-nA9BE-nABEA-nABFC-nAC71-nACE1-nADA3-nAE4D-nAFD6-nB023-nB03A-nB155-nB1DC-nB21A-nB266-nB277-nB27B-nB285-nB2F3-nB331-nB38E-nB3B1-nB3EE-nB457-nB513-nB557-nB5FA-nB6A3-nB6C1-nB750-nB7D9-nB7E3-nB812-nB820-nB8B5-nB93D-nB9D3-nBA33-nBA99-nBA9B-nBB3A-nBB56-nBBCA-nBC2A-nBC9B-nBD31-nBE32-nBF00-nBF2C-nC115-nC25A-nC277-nC2DA-nC2DD-nC372-nC4BF-nC511-nC6E8-nC902-nC904-nC946-nC9E0-nCA92-nCB06-nCBEE-nCCC8-nCE0B-nCE77-nCF0B-nD04C-nD1B5-nD1CF-nD1D0-nD519-nD586-nD65F-nD6A1-nD6EB-nD83D-nD8FE-nD915-nD993-nD9B9-nDA2F-nDAE9-nDC09-nDC14-nDC25-nDD57-nDDF8-nDF01-nDFA1-nDFA6-nDFF8-nE03A-nE0E1-nE14E-nE153-nE169-nE1B0-nE1E7-nE1F9-nE305-nE39F-nE3D0-nE3D3-nE4C6-nE4D6-nE593-nE60B-nE647-nE69F-nE6AD-nE6BA-nE6C5-nE799-nE8A3-nE8AE-nE8EF-nE9DB-nEB26-nEB65-nEC6E-nECC8-nECF1-nED93-nEDFA-nEE30-nEE60-nEECC-nEF6A-nEF7D-nF011-nF017-nF01C-nF050-nF06A-nF092-nF0BD-nF0CA-nF0F7-nF19B-nF2B8-nF442-nF5EB-nF708-nF724-nF728-nF74B-nF74E-nF7DA-nF845-nF89E-nF8E7-nF8E8-nF953-nF95E-nFA3D-nFAD5-nFBC9-nFCED-nFD2C-nFE6E-nFF8E-nFFA3-nFFEA

\subsection{Estrategia}

Se utiliza un arreglo "a" para guardar todos los dígitos del archivo tex y "hexarray" (solo con ceros y 10000 en espacio de memoria) para guardar las secuencias de 4 dígitos hexadecimeales.
Nota: Para trabajar con números hexadecimales se utilizó la función "strtol(char, NULL, 16)". 

Luego se recorrió el arreglo "hexarray" ordenando todas las secuencias de 4 dígitos con el método de ordenamiento por inserción.

Una vez ordenado de menor a mayor, el arreglo es recorrido y se van mostrando las secuencias que se repiten al menos 1 vez con su respectiva frecuencia.

\subsection{Justificación}

Se decide crear una lista ordenada con todos los números de 4 dígitos del archivo tex debido a su facilidad de uso y porque el coste en memoria de este arreglo es de n-4. La complejidad de obtener todos los números de 4 dígitos del arreglo con los datos del archivo es de n-4.

Se decide usar el método de inserción dado a sus bajos requerimientos de memoria y estabilidad (nunca intercambia registros con claves iguales). Pese a su lento proceso y la numerosas comparaciones necesarias a realizar, muestra un comportamiento razonablemente bien en gran cantidad de situaciones, y esta no es la excepción, además que ser de fácil implementación.

El método de inserción tiene como complejidad en el peor de los casos O($n^2$) y en el mejor de los casos O(n) (o sea, cuando el arreglo está ordenado), y el recorrido guardar las secuencias de 4 dígitos hexadecimeales una es de n-4.

Por lo que la complejidad del algoritmo sería: 

En el mejor caso: n + (n-4)
\\
En el peor caso: $n^2$ + (n-4)


Y su uso de memoria total es de 2n-4, debido a que n es lo que cuesta el arreglo con los dígitos de un archivo, n-4 el arreglo con todos los números de 4 dígitos y el insertion sort solo requiere O(1) de espacio adicional.


\newpage
\section{Problema 3}

\subsection{Solución Encontrada}

$Archivo_3.tex$ no arroja ningún resultado, por lo que no hay ninguna secuencia de 6 dígitos repetidas según el corte que se le dio.

\subsection{Estrategia}


Se uso un método similar al segundo ejercicio, se guarda la data del texto $Archivo_3.tex$ en un arreglo A de largo n, este arreglo se recorre $\frac{n}{6}$ veces extrayendo en segmentos de 6 dígitos cada número del arreglo y guardandolos en un nuevo arreglo B en forma de INT.

Este arreglo B se ordena usando el método de ordenamiento quicksort, de esta forma el arreglo B tiene los elementos repetidos uno al lado del otro, por último se recorre el arreglo contando la frecuencia con la que aparece cada elemento repetido de manera seguida y los que tengan frecuencia mayor a uno se muestran.


\subsection{Justificación}

Se utilizo este método debido a que su memoria es lineal, solo ocupa dos arreglos de largo n y $\frac{n}{6}$, el quicksort ocupa $\log n$ de memoria, por lo que ocupa $\frac{7n}{6} + \log n$ de memoria.

Además de su rapidez, dividir el arreglo en secciones de 6 tiene complejidad de $\frac{n}{6}$ y ordenar el arreglo con el método quicksort tiene una complejidad de $n \log n $, por lo que su complejidad final es de $\frac{n}{6} + n \log n $, haciendo que el algoritmo sea de complejidad lineal logarítmica.

Se usa el algoritmo quicksort debido a su rapidez y que ser el largo del arreglo par es más seguro que el pivote elegido sea el correcto.


\end{document}